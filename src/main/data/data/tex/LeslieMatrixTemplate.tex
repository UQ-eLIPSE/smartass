%Simple question on the usage of Leslie matrix. 
%DEFINES THE PREAMBLE FOR STANDARD SMARTASS ASSIGNMENT

\documentclass[10pt,fleqn]{article}

%\usepackage{epsf}
\usepackage{amsmath}
\usepackage{amssymb} 
\usepackage{cancel} %cancelled numbers 
%\usepackage{lcg} %random numbers generator
\usepackage{ifthen} %logical constructs
<<<<<<< HEAD
=======
\usepackage{enumerate}
\usepackage{siunitx}

>>>>>>> Added additional default LaTeX packages.

\newif\ifpdf
\ifx\pdfoutput\undefined
\pdffalse
\else
\pdfoutput=1
\pdftrue
\fi

\ifpdf
\usepackage[pdftex]{overpic}
\usepackage[pdftex]{graphicx}
\pdfcompresslevel=9
\pdfpagewidth=215truemm 
\pdfpageheight=297truemm
\pdfhorigin=1truein
\pdfvorigin=1truein
\else
\usepackage{overpic}
\usepackage{graphicx}
\fi

\setlength{\textwidth}{185.0mm}
\setlength{\textheight}{660pt}
\setlength{\oddsidemargin}{-15.0mm}
\setlength{\evensidemargin}{-15.0mm}
\setlength{\headsep}{0pt}
%\setlength{\topmargin}{-29.0truemm}
\setlength{\parindent}{0.0truemm}

\renewcommand {\theenumi}{\bf \arabic{enumi}}
\renewcommand {\theenumii} {\textbf{\arabic {enumii}}}
\renewcommand {\theenumiii} {\textbf {\roman{enumiii}}}


\begin{document}
%%BEGIN DEF 
%definition of variables/modules used
%/**
%* Constructor LeslieMatrixModule initialises the question
%* with parameters passing.
%* In case of "values" as a first parameter:
%* @params  params[0] - string "values", indicates types of parameters passing,
%*		params[1] - number of age groups,
%*          params[2] - number of generations,
%*          params[from 3 to number number_of_ages+2] - the numbers of individuals in each age group in initial population,
%*          params[from number_of_ages+3 to 2number_of_ages+2] - the fertility of the corresponding age groups,
%*          params[from 2number_of_ages+3 to 3number_of_ages+1] - the fraction of individuals that survives from the corresponding age group,   
% for instance, v1(values,3,5,10,0,0,0,4,1,0.8,0.9);
%* or
%@params  params[0] - string "limits", indicating that there are limits being passed
%*		params[1] - number of age groups,
%*          parmas[2-3] - min and max limits for number of generations (including initial generation) required, 
%*          params[from 4 to 2number_of_ages+3] - limits for numbers of individuals in each age group in initial population,
%*		params[from 2number_of_ages+4 to 4number_of_ages+3] - limits for fertility of each age group,
%*          params[from 4number_of_ages+4 to 6number_of_ages+1] - limits for the fraction of individuals that survives from the corresponding age group.
%v1(limits,2,2,8,5,10,0,1,0,1,10,15,0.1,0.3);
#<
LeslieMatrixModule v1;
#>
%%END DEF

%%BEGIN QUESTION
A particular organism's population is modelled using a simple Leslie model with #<v1.number_of_ages#> life stages.

The fertility of each life stage is:\; #<v1.fecundity_string#>.\\
The survival rate from each life stage to the next is:\; #<v1.survival_string#>.\\ 
The initial population is:\; #<v1.initial_population_string#>.

Find the Leslie matrix $L$ and initial population vector $P_0$, then
estimate the population at times $t=1$ to $t=#<v1.number_of_steps#>$.
(Round your answers to 1 decimal place at each time step.)
%%END QUESTION

%%BEGIN SOLUTION

The initial population vector $P_0$ and the Leslie matrix $L$ are: 
$$P_0=#<v1.population0#>\quad\mbox{and}\quad
L=#<v1.leslie_matrix#>.$$\\
Then to find the population at time step $t+1$ we calculate $P_{t+1}=L\times P_t$, as follows:\\
#<v1.working#>\\

%%END SOLUTION


%%BEGIN SHORTANSWER

The initial population vector $P_0$, the Leslie matrix $L$, and the final
population vector $P_{#<v1.number_of_steps#>}$ are: 
$$P_0=#<v1.population0#>\quad\mbox{and}\quad
L=#<v1.leslie_matrix#>\quad\mbox{and}\quad
P_{#<v1.number_of_steps#>}=#<v1.last_population_vector#>.$$

%%END SHORTANSWER
\end{document}
