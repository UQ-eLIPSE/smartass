%Find the equation of the line parallel to ay+bx+c=0 and passing through the point (x1, y1)
%Unlike LinearPrallelTemplate, this question does not allow more than 3 non-zero terms in the original equation.
%DEFINES THE PREAMBLE FOR STANDARD SMARTASS ASSIGNMENT

\documentclass[10pt,fleqn]{article}

%\usepackage{epsf}
\usepackage{amsmath}
\usepackage{amssymb} 
\usepackage{cancel} %cancelled numbers 
%\usepackage{lcg} %random numbers generator
\usepackage{ifthen} %logical constructs
<<<<<<< HEAD
=======
\usepackage{enumerate}
\usepackage{siunitx}

>>>>>>> Added additional default LaTeX packages.

\newif\ifpdf
\ifx\pdfoutput\undefined
\pdffalse
\else
\pdfoutput=1
\pdftrue
\fi

\ifpdf
\usepackage[pdftex]{overpic}
\usepackage[pdftex]{graphicx}
\pdfcompresslevel=9
\pdfpagewidth=215truemm 
\pdfpageheight=297truemm
\pdfhorigin=1truein
\pdfvorigin=1truein
\else
\usepackage{overpic}
\usepackage{graphicx}
\fi

\setlength{\textwidth}{185.0mm}
\setlength{\textheight}{660pt}
\setlength{\oddsidemargin}{-15.0mm}
\setlength{\evensidemargin}{-15.0mm}
\setlength{\headsep}{0pt}
%\setlength{\topmargin}{-29.0truemm}
\setlength{\parindent}{0.0truemm}

\renewcommand {\theenumi}{\bf \arabic{enumi}}
\renewcommand {\theenumii} {\textbf{\arabic {enumii}}}
\renewcommand {\theenumiii} {\textbf {\roman{enumiii}}}


\begin{document}
%definition of variables/modules used
%Possible parameters ParallelLineSimpleModule(horizontal) - module will generate horizontal lines
%Do not use parametre "vertical" with this template as text of template does not match to output and gradient m 
%is not defined for vertical lines. See LineParallelToVerticalTemplate.tex
%%BEGIN DEF 
#<
ParallelLineSimpleModule v1;
#>
%%END DEF

%%BEGIN QUESTION

Find the equation of the line parallel to \hspace{2mm} #<v1.originalequation#> \hspace{2mm} and 
passing through the point $(#<v1.x1#>,#<v1.y1#>)$.

%%END QUESTION

%%BEGIN SOLUTION

To find the equation of the new line, we first need the gradient of the original line. Now, 
#<v1.rewriteoriginalequation#>
Hence, the gradient of the original line is  $m=#<v1.m#>$. \\
The new line is parallel to the original line, so it has the same gradient as the original line. Thus the equation
of the line is \hspace{2mm} #<v1.newequation#> \hspace{2mm} and we can substitute the coordinates of the point
 $(x_{1}, y_{1})=(#<v1.x1#>,#<v1.y1#>)$ into this equation to get the value for $c$.\\
#<v1.findc#>\\
Hence the equation of the line is  \hspace{2mm} $#<v1.equation#>$.

%%END SOLUTION

%%BEGIN SHORTANSWER

$#<v1.equation#>$

%%END SHORTANSWER

\end{document}
