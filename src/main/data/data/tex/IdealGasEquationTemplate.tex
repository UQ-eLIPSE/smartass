%Simple question on ideal gas equation p1v1/t1=p2v2/t2. Given an initial volume of the gas v1, initial pressure p1, 
%and initial temperature t1, find the volume v2 (temperature, pressure) if two other parameters changed (and given). 
%DEFINES THE PREAMBLE FOR STANDARD SMARTASS ASSIGNMENT

\documentclass[10pt,fleqn]{article}

%\usepackage{epsf}
\usepackage{amsmath}
\usepackage{amssymb} 
\usepackage{cancel} %cancelled numbers 
%\usepackage{lcg} %random numbers generator
\usepackage{ifthen} %logical constructs
<<<<<<< HEAD
=======
\usepackage{enumerate}
\usepackage{siunitx}

>>>>>>> Added additional default LaTeX packages.

\newif\ifpdf
\ifx\pdfoutput\undefined
\pdffalse
\else
\pdfoutput=1
\pdftrue
\fi

\ifpdf
\usepackage[pdftex]{overpic}
\usepackage[pdftex]{graphicx}
\pdfcompresslevel=9
\pdfpagewidth=215truemm 
\pdfpageheight=297truemm
\pdfhorigin=1truein
\pdfvorigin=1truein
\else
\usepackage{overpic}
\usepackage{graphicx}
\fi

\setlength{\textwidth}{185.0mm}
\setlength{\textheight}{660pt}
\setlength{\oddsidemargin}{-15.0mm}
\setlength{\evensidemargin}{-15.0mm}
\setlength{\headsep}{0pt}
%\setlength{\topmargin}{-29.0truemm}
\setlength{\parindent}{0.0truemm}

\renewcommand {\theenumi}{\bf \arabic{enumi}}
\renewcommand {\theenumii} {\textbf{\arabic {enumii}}}
\renewcommand {\theenumiii} {\textbf {\roman{enumiii}}}


\begin{document}
%%BEGIN DEF 
%definition of variables/modules used
%parameters IdealGasEquationModule vi(pvt, MIN_P, MAX_P, MIN_V, MAX_V, MIN_T, MAX_T); where
%In case of 7 parameters (limits are set by passing parameters):
%	   pvt indicates which parameter is unknown, if "p" find pressure, "v" - find volume, "t" - find temperature
%	   MIN_P - minimum pressure, double value, 
%        MAX_P - maximum pressure, double
%	   MIN_V - minimum volume, double x 10^-3
%        MAX_V - maximum volume, double
%        MIN_T - minimum temperature in K,
%        MAX_T - maximum temperature	   
%
% or In case of 6 parameters (everything is set by passing parameters):
%        pvt indicates which parameter is unknown if "p" find pressure p2, "v" - find volume, "t" - find temperature
%	   p1 - initial pressure,
%	   v1 - initial volume, times 10^-3
%        t1 - initial temperature in K,
%        p2 or v2 (depending on pvt passed),
%        v2 or t2 (depending on pvt passed).
%vi(t,45.8,3.48,258,62.2,3.07);
#<
IdealGasEquationModule vi;
#>
%%END DEF

%%BEGIN QUESTION

A sample of diborane gas (B$_{2}$H$_{6}$) has a pressure of $#<vi.p1#>$ kPa at a temperature of $#<vi.t1c#>^{\circ}$C and
a volume of $#<vi.v1#>$ m$^{3}$. If conditions are changed such that the
\if p#<vi.pvt#>
temperature is $#<vi.t2c#>^{\circ}$C and the volume is $#<vi.v2#>$ m$^{3}$, what will be the pressure of the sample? 
\fi
\if v#<vi.pvt#>
temperature is $#<vi.t2c#>^{\circ}$C and the pressure is $#<vi.p2#>$ kPa, what will be the volume of the sample? 
\fi
\if t#<vi.pvt#>
pressure is $#<vi.p2#>$ kPa and the volume is $#<vi.v2#>$ m$^{3}$, what will be the temperature of the sample? 
\fi

%%END QUESTION

%%BEGIN SOLUTION

The number of moles of the gas remains constant, so $\dfrac{P_{1}V_{1}}{T_{1}}=\dfrac{P_{2}V_{2}}{T_{2}}$ \\
$\therefore$ 
\if v#<vi.pvt#>
$V_{2}=\dfrac{P_{1}V_{1}}{T_{1}}\times \dfrac{T_{2}}{P_{2}}$ 

Here,
\begin{alignat*}{3}
&P_{1}=#<vi.p1#> \mbox{kPa}& \hspace{30mm} &P_{2}=#<vi.p2#> \mbox{kPa}\\
&T_{1}=#<vi.t1c#>^{\circ}\mbox{C}+273=#<vi.t1k#> \mbox{K}& \hspace{30mm} &T_{2}=#<vi.t2c#>^{\circ}\mbox{C}+273=#<vi.t2k#> \mbox{K}\\ 
&V_{1}=#<vi.v1#> \mbox{m}^{3}&  \hspace{30mm} &V_{2}= ?
\end{alignat*}
\vspace{1mm}
So $V_{2}=\dfrac{#<vi.p1#> \mbox{kPa}\cdot #<vi.v1#> \mbox{m}^{3} \cdot #<vi.t2k#> \mbox{K}}
{#<vi.t1k#> \mbox{K} \cdot #<vi.p2#> \mbox{kPa}} = #<vi.v2#> \mbox{m}^{3}$ \\  
\fi
\if p#<vi.pvt#>
$P_{2}=\dfrac{P_{1}V_{1}}{T_{1}}\times \dfrac{T_{2}}{V_{2}}$ 

Here,
\begin{alignat*}{3}
&V_{1}=#<vi.v1#> \mbox{m}^{3}&  \hspace{30mm} &V_{2}=#<vi.v2#> \mbox{m}^{3}\\ 
&T_{1}=#<vi.t1c#>^{\circ}\mbox{C}+273=#<vi.t1k#> \mbox{K}& \hspace{30mm} &T_{2}=#<vi.t2c#>^{\circ}\mbox{C}+273=#<vi.t2k#> \mbox{K}\\ 
&P_{1}=#<vi.p1#> \mbox{kPa}& \hspace{30mm} &P_{2}= ?
\end{alignat*}
\vspace{1mm}
So $P_{2}=\dfrac{#<vi.p1#> \mbox{kPa}\cdot #<vi.v1#> \mbox{m}^{3} \cdot #<vi.t2k#> \mbox{K}}
{#<vi.t1k#>\mbox{K} \cdot #<vi.v2#> \mbox{m}^{3}} = #<vi.p2#> \mbox{kPa}$ \\  
\fi
\if t#<vi.pvt#>
$T_{2}=P_{2}V_{2} \times \dfrac{T_{1}}{P_{1}V_{1}}$ 

Here,
\begin{alignat*}{3}
&P_{1}=#<vi.p1#> \mbox{kPa}& \hspace{30mm} &P_{2}=#<vi.p2#> \mbox{kPa}\\
&V_{1}=#<vi.v1#> \mbox{m}^{3}&  \hspace{30mm} &V_{2}=#<vi.v2#> \mbox{m}^{3}\\ 
&T_{1}=#<vi.t1c#>^{\circ}\mbox{C}+273=#<vi.t1k#> \mbox{K}& \hspace{30mm} &T_{2}= ?
\end{alignat*}
\vspace{1mm}
So $T_{2}=\dfrac{#<vi.p2#> \mbox{kPa}\cdot #<vi.v2#> \mbox{m}^{3} \cdot #<vi.t1k#> \mbox{K}}
{#<vi.p1#> \mbox{kPa}\cdot #<vi.v1#> \mbox{m}^{3}} = #<vi.t2k#> \mbox{K}$\\[2.4mm]
and converting this to Celsius gives $T_{2}=#<vi.t2k#> \mbox{K}-273=#<vi.t2c#>^{\circ}\mbox{C}$  
\fi

%%END SOLUTION

%%BEGIN SHORTANSWER
\if v#<vi.pvt#>
$#<vi.v2#> \mbox{m}^{3}$ 
\fi
\if p#<vi.pvt#>
$#<vi.p2#> \mbox{kPa}$ 
\fi
\if t#<vi.pvt#>
$#<vi.t2c#>^{\circ}\mbox{C}$  
\fi
%%END SHORTANSWER
\end{document}
