%Solve each of the following four equations without using the quadratic formula. Equations are of the form:
%  1) ax(bx+c) = 0
%  2) (ax+b)(cx+d)= 0
%  3) e(ax+b)(cx+d)= 0
%  4) (ax+x)^k = 0 (actually, this is not quadratic one) 
%DEFINES THE PREAMBLE FOR STANDARD SMARTASS ASSIGNMENT

\documentclass[10pt,fleqn]{article}

%\usepackage{epsf}
\usepackage{amsmath}
\usepackage{amssymb} 
\usepackage{cancel} %cancelled numbers 
%\usepackage{lcg} %random numbers generator
\usepackage{ifthen} %logical constructs
<<<<<<< HEAD
=======
\usepackage{enumerate}
\usepackage{siunitx}

>>>>>>> Added additional default LaTeX packages.

\newif\ifpdf
\ifx\pdfoutput\undefined
\pdffalse
\else
\pdfoutput=1
\pdftrue
\fi

\ifpdf
\usepackage[pdftex]{overpic}
\usepackage[pdftex]{graphicx}
\pdfcompresslevel=9
\pdfpagewidth=215truemm 
\pdfpageheight=297truemm
\pdfhorigin=1truein
\pdfvorigin=1truein
\else
\usepackage{overpic}
\usepackage{graphicx}
\fi

\setlength{\textwidth}{185.0mm}
\setlength{\textheight}{660pt}
\setlength{\oddsidemargin}{-15.0mm}
\setlength{\evensidemargin}{-15.0mm}
\setlength{\headsep}{0pt}
%\setlength{\topmargin}{-29.0truemm}
\setlength{\parindent}{0.0truemm}

\renewcommand {\theenumi}{\bf \arabic{enumi}}
\renewcommand {\theenumii} {\textbf{\arabic {enumii}}}
\renewcommand {\theenumiii} {\textbf {\roman{enumiii}}}


\begin{document}
%%BEGIN DEF 
%definition of variables/modules used
%Possible parameter - type of question (see above)
%QuadraticNoFormulaModule v1(2); would generate (ax+b)(cx+d)=0 equation
#<
QuadraticNoFormulaModule v1(1);
QuadraticNoFormulaModule v2(2);
QuadraticNoFormulaModule v3(3);
QuadraticNoFormulaModule v4(4);
#>
%%END DEF

%%BEGIN QUESTION

Solve each of the following equations {\bf without} using the quadratic formula:
\begin{enumerate}
\item
#<v1.question#>
\item
#<v2.question#>
\item
#<v3.question#>
\item
#<v4.question#>
\end{enumerate}

%%END QUESTION

%%BEGIN SOLUTION

To solve each of these, remember that if $a \times b=0$ , then either $a=0$ or $b=0$;  
and also that $0^{n}=0$ for any natural number $n$. Then:
\begin{enumerate}
\item
#<v1.question#>,  so
#<v1.solution#>
\item
#<v2.question#>, so
#<v2.solution#>
\item
#<v3.question#>, so
#<v3.solution#>
\item
#<v4.question#>, so
#<v4.solution#>
\end{enumerate}

%%END SOLUTION

%%BEGIN SHORTANSWER
\begin{enumerate}
\item
#<v1.shortanswer#>
\item
#<v2.shortanswer#>
\item
#<v3.shortanswer#>
\item
#<v4.shortanswer#>

\end{enumerate}
%%END SHORTANSWER

\end{document}
