%Solve a physics floating problem.
%Template 
%DEFINES THE PREAMBLE FOR STANDARD SMARTASS ASSIGNMENT

\documentclass[10pt,fleqn]{article}

%\usepackage{epsf}
\usepackage{amsmath}
\usepackage{amssymb} 
\usepackage{cancel} %cancelled numbers 
%\usepackage{lcg} %random numbers generator
\usepackage{ifthen} %logical constructs
<<<<<<< HEAD
=======
\usepackage{enumerate}
\usepackage{siunitx}

>>>>>>> Added additional default LaTeX packages.

\newif\ifpdf
\ifx\pdfoutput\undefined
\pdffalse
\else
\pdfoutput=1
\pdftrue
\fi

\ifpdf
\usepackage[pdftex]{overpic}
\usepackage[pdftex]{graphicx}
\pdfcompresslevel=9
\pdfpagewidth=215truemm 
\pdfpageheight=297truemm
\pdfhorigin=1truein
\pdfvorigin=1truein
\else
\usepackage{overpic}
\usepackage{graphicx}
\fi

\setlength{\textwidth}{185.0mm}
\setlength{\textheight}{660pt}
\setlength{\oddsidemargin}{-15.0mm}
\setlength{\evensidemargin}{-15.0mm}
\setlength{\headsep}{0pt}
%\setlength{\topmargin}{-29.0truemm}
\setlength{\parindent}{0.0truemm}

\renewcommand {\theenumi}{\bf \arabic{enumi}}
\renewcommand {\theenumii} {\textbf{\arabic {enumii}}}
\renewcommand {\theenumiii} {\textbf {\roman{enumiii}}}


\begin{document}


%\ifx \@testmacro \@empty
%Code to be executed if the macro is undefined
%\else
%Code to be executed if the macro IS defined


\providecommand{\surfboardtemplatecommands}{defined}
\providecommand{\kg}{\ensuremath{\text{kg}}}
\providecommand{\metre}{\ensuremath{\text{m}}}
\providecommand{\volume}{\ensuremath{\metre^3}}
\providecommand{\density}{\ensuremath{\kg/\volume}}

%%BEGIN DEF 
%definition of variables/modules used
%possible parameters:
% params[0] - string representing integer density of the block material,
% params[1] - mass of the person,
% params[2] - part of the block, that will remain underwater,
% params[3] (optional) - water density - double number.
%for instance, for the person of 
%if there are no parameters - smartass will generate them randomly
#<
SurfboardModule v1;
#>
%%END DEF


%%BEGIN QUESTION
\begin{enumerate}

\item Ben wants to build his own surfboard.  He is going to use foam
  that has a density of #<v1.blockdensity#> \density.  Using
  Archimede's principle, what's the minimum mass of foam required to
  \emph{two decimal places} if Ben has a mass of #<v1.objectmass#>
  \kg{} (i.e. the foam is just at the surface of the water: remove any
  foam and the block will sink)?

\item Ben has worked out that for his surfboard to work properly, he
  has to have at least #<v1.percentabovewaterrandom#>\% of the block
  above water.  Using the same requirements as above, how much foam
  does he need now?

\end{enumerate}

Hints: 
\begin{itemize}
\item $1 \volume$ of foam will displace $1 \volume$ of water.
\item $1 \kg$ of water can support $1 \kg$ of another object.
\item Water has a density of #<v1.liquiddensity#> \density.
\item Ignore the shape of the board: consider the foam to be just a
  solid block.  However, don't forget to take the mass of the block
  into account!
\end{itemize}

%%END QUESTION

%%BEGIN SOLUTION

\begin{enumerate}

\item Let us denote the mass and volume of the block as $b$ and $V$
  respectively, the mass of Ben as $m$, the density of water as
  $\rho_w$ and the density of the block as $\rho_b$.  The mass of
  water $w$ displaced by the block is then:

  \begin{eqnarray*}
    w & = & \rho_w \times V\\
      & = & \rho_w \times \dfrac{b}{\rho_b}\\
      & = & \dfrac{\rho_w}{\rho_b} \times b
  \end{eqnarray*}

  The minimum amount of foam needed will be that which keeps the
  system in equilibrium (i.e. any more and the block will rise, any
  less and it will sink).  Thus, the mass of water displaced must
  equal Ben's mass \emph{as well as the mass of the block!}.

  \begin{equation*}
    \begin{array}{lrcl}
      & w & = & m + b\\
      \Longrightarrow & \dfrac{\rho_w}{\rho_b} \times b & = & m + b\\
      \Longrightarrow & \dfrac{\rho_w}{\rho_b} \times b - b & = & m\\
      \Longrightarrow & \left(\dfrac{\rho_w}{\rho_b} - 1\right) \times b 
      & = & m\\
      \Longrightarrow & b & = & \dfrac{m}{\dfrac{\rho_w}{\rho_b} - 1}
    \end{array}
  \end{equation*}

  Substituting in the values and rounding off to two decimal places,
  we have:

  \begin{eqnarray*}
    b & = & \dfrac{#<v1.objectmass#>}{ \dfrac{ #<v1.liquiddensity#> }{%
        \;#<v1.blockdensity#> } - 1\;}\\
      & = & #<v1.blockmassall#> \kg
  \end{eqnarray*}

Thus, Ben will need at least $#<v1.blockmassall#> \kg$ of foam to
support his weight.

\item Using the same symbols as above, let us further denote the
  percentage of the block that is \emph{underwater} by $p$.  That is:

  \begin{equation*}
    p = 100 \% - #<v1.percentabovewaterrandom#> \% 
    = #<v1.percentunderwaterrandom#> \% 
    = #<v1.decimalunderwaterrandom#>
  \end{equation*}


  The mass of water that is displaced is now thus dependent upon how
  much of the block is underwater, $V'$, where:

  \begin{equation*}
    V' = p \times V
  \end{equation*}

  Thus, the mass of water $w'$ displaced with this reduced volume will
  be:

  \begin{eqnarray*}
    w' & = & \rho_w \times V'\\
       & = & \rho_w \times \left(p \times V\right)\\
       & = & p \rho_w \times \dfrac{b}{\rho_b}\\
       & = & \dfrac{p \rho_w}{\rho_b} \times b
  \end{eqnarray*}

  Again, we want the block to be in equilibrium, so:

  \begin{displaymath}
    \begin{array}{lrcl}
      & w' & = & m + b\\
      \Longrightarrow & \dfrac{p \rho_w}{\rho_b} \times b & = & m + b\\
      \Longrightarrow & \dfrac{p \rho_w}{\rho_b} \times b - b & = & m\\
      \Longrightarrow & \left(\dfrac{p \rho_w}{\rho_b} - 1\right) \times b 
      & = &  m\\
      \Longrightarrow & b & = & \dfrac{m}{\dfrac{p \rho_w}{\rho_b} - 1}
    \end{array}
  \end{displaymath}

  Substituting in the values and rounding off to two decimal places,
  we have:

  \begin{eqnarray*}
    b & = & \dfrac{#<v1.objectmass#>}{\dfrac{ #<v1.decimalunderwaterrandom#>
        \times #<v1.liquiddensity#> }{\;#<v1.blockdensity#> } - 1\;}\\
      & = & #<v1.blockmassrandom#> \kg
  \end{eqnarray*}

Thus, Ben will need at least $#<v1.blockmassrandom#> \kg$ of foam to
support his weight if he wants to ensure that at least
$#<v1.percentabovewaterrandom#> \%$ of it remains above water.

\end{enumerate}

%%END SOLUTION

%%BEGIN SHORTANSWER
\begin{enumerate}
\item #<v1.blockmassall#> \kg
\item #<v1.blockmassrandom#> \kg
\end{enumerate}
%%END SHORTANSWER

\end{document}
