%Write a program investigating the Collatz Conjecture
%DEFINES THE PREAMBLE FOR STANDARD SMARTASS ASSIGNMENT

\documentclass[10pt,fleqn]{article}

%\usepackage{epsf}
\usepackage{amsmath}
\usepackage{amssymb} 
\usepackage{cancel} %cancelled numbers 
%\usepackage{lcg} %random numbers generator
\usepackage{ifthen} %logical constructs
<<<<<<< HEAD
=======
\usepackage{enumerate}
\usepackage{siunitx}

>>>>>>> Added additional default LaTeX packages.

\newif\ifpdf
\ifx\pdfoutput\undefined
\pdffalse
\else
\pdfoutput=1
\pdftrue
\fi

\ifpdf
\usepackage[pdftex]{overpic}
\usepackage[pdftex]{graphicx}
\pdfcompresslevel=9
\pdfpagewidth=215truemm 
\pdfpageheight=297truemm
\pdfhorigin=1truein
\pdfvorigin=1truein
\else
\usepackage{overpic}
\usepackage{graphicx}
\fi

\setlength{\textwidth}{185.0mm}
\setlength{\textheight}{660pt}
\setlength{\oddsidemargin}{-15.0mm}
\setlength{\evensidemargin}{-15.0mm}
\setlength{\headsep}{0pt}
%\setlength{\topmargin}{-29.0truemm}
\setlength{\parindent}{0.0truemm}

\renewcommand {\theenumi}{\bf \arabic{enumi}}
\renewcommand {\theenumii} {\textbf{\arabic {enumii}}}
\renewcommand {\theenumiii} {\textbf {\roman{enumiii}}}


\begin{document}
%%BEGIN DEF
%definition of variables/modules used
#<
CollatzModule v1;
#>
%%END DEF

%%BEGIN QUESTION
In the vain hope of achieving everlasting immortality (at least
amongst mathematicians), Peter has made his own Collatz-style
conjecture:

Consider an arbitrary integer #<v1.var#>.
\begin{itemize}
\item If #<v1.var#> is divisible by #<v1.mod#>, then divide it by #<v1.mod#>.
\item Otherwise, assign #<v1.var#> the value of #<v1.expr#>
\end{itemize}

Peter proposes that -- just like with the Collatz Conjecture -- if you
start with any arbitrary positive integer and repeatedly keep applying
the above rule, then you will eventually reach 1.

Write a Python program to experimentally investigate this by
implementing the above rule as a function \emph{#<v1.func#>}, then
starting with an initial value of #<v1.start#> use a loop to
repeatedly apply the function on the result, printing out the value at
each iteration of the loop.  In case Peter's conjecture doesn't work,
you should ensure that the loop iterates no more than #<v1.max#>
times.  Finally, print a message stating whether or not a value of 1
was reached, and how many iterations of the loop it took (if the value
of 1 wasn't reached, then say that).

%%END QUESTION

%%BEGIN SOLUTION
The full Python inputs and results are shown below.  #<v1.resultwarning#>

#<v1.code#>

%%END SOLUTION

%%BEGIN SHORTANSWER
#<v1.resultwarning#>

#<v1.answer#>

%%END SHORTANSWER

\end{document}
