%Intersection, union and set differences of three sets
%DEFINES THE PREAMBLE FOR STANDARD SMARTASS ASSIGNMENT

\documentclass[10pt,fleqn]{article}

%\usepackage{epsf}
\usepackage{amsmath}
\usepackage{amssymb} 
\usepackage{cancel} %cancelled numbers 
%\usepackage{lcg} %random numbers generator
\usepackage{ifthen} %logical constructs
<<<<<<< HEAD
=======
\usepackage{enumerate}
\usepackage{siunitx}

>>>>>>> Added additional default LaTeX packages.

\newif\ifpdf
\ifx\pdfoutput\undefined
\pdffalse
\else
\pdfoutput=1
\pdftrue
\fi

\ifpdf
\usepackage[pdftex]{overpic}
\usepackage[pdftex]{graphicx}
\pdfcompresslevel=9
\pdfpagewidth=215truemm 
\pdfpageheight=297truemm
\pdfhorigin=1truein
\pdfvorigin=1truein
\else
\usepackage{overpic}
\usepackage{graphicx}
\fi

\setlength{\textwidth}{185.0mm}
\setlength{\textheight}{660pt}
\setlength{\oddsidemargin}{-15.0mm}
\setlength{\evensidemargin}{-15.0mm}
\setlength{\headsep}{0pt}
%\setlength{\topmargin}{-29.0truemm}
\setlength{\parindent}{0.0truemm}

\renewcommand {\theenumi}{\bf \arabic{enumi}}
\renewcommand {\theenumii} {\textbf{\arabic {enumii}}}
\renewcommand {\theenumiii} {\textbf {\roman{enumiii}}}


\begin{document}
%%BEGIN DEF 
%definition of variables/modules used

#<
OperationsOnThreeSetsModule v1;
#>
%Parameteres for OperationsOnTwoSetsModule: (name of first set, name of second set, name of third set, elements of first set in {} 
%separated by comma, elements of second set in {} separated by comma, elements of third set in {} separated by comma, for instance:
%A, B, C, { -3,-2,-1, 0, 1, 2, 3, 4, 5, 6},{-3, -2, 9, 10, 11, 12, 13, 14, 15, 16}, {1, 2, 3});

%%END DEF

%%BEGIN QUESTION
For the following questions let \hspace{3mm} #<v1.name1#>=#<v1.setrepresentation1#>, #<v1.name2#>=#<v1.setrepresentation2#> 
, #<v1.name3#>=#<v1.setrepresentation3#>
\begin{enumerate}
\item
Write down the elements of set #<v1.question1#> .
\item
Write down the elements of the set #<v1.question2#> .
\item
Write down the elements of the set #<v1.question3#> .
\item
Write down the elements of the set #<v1.question4#> .
\item
Write down the elements of the set #<v1.question5#> . Shade the corresponding region on the Venn diagram.
\item
Write down the elements of the set #<v1.question6#> .
\item
Write down the elements of the set #<v1.question7#> .
\item
Write down the elements of the set #<v1.question8#> .
\item
Write down the elements of the set #<v1.question9#> .


\end{enumerate}

%%END QUESTION

%%BEGIN SOLUTION
\begin{enumerate}
\item
#<v1.solution1#> \\
\item
#<v1.solution2#> \\
\item
#<v1.solution3#> \\
\item
#<v1.solution4#> \\
\item
#<v1.solution5#>\\
%Venn diagram here:
\begin{overpic}[scale=1]
{ThreeSetsOps#<v1.question5diagram#>.pdf}
\put(-10,65){\huge $#<v1.name1#>$}
\put(103,65){\huge $#<v1.name2#>$}
\put(50,-10){\huge $#<v1.name3#>$}
\end{overpic}\\
\vspace{5mm}

\item
#<v1.solution6#> \\
\item
#<v1.solution7#> \\
\item
#<v1.solution8#> \\
\item
#<v1.solution9#> 

\end{enumerate}
%%END SOLUTION

%%BEGIN SHORTANSWER
\begin{enumerate}
\item
#<v1.shortanswer1#>
\item
#<v1.shortanswer2#>
\item
#<v1.shortanswer3#>
\item
#<v1.shortanswer4#>
\item
#<v1.shortanswer5#>
\item
#<v1.shortanswer6#>
\item
#<v1.shortanswer7#>
\item
#<v1.shortanswer8#> 
\item
#<v1.shortanswer9#> 

\end{enumerate}

%%END SHORTANSWER

\end{document}
