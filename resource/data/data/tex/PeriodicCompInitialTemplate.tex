%Choose initial value to invest in periodic compounding interest example.
%DEFINES THE PREAMBLE FOR STANDARD SMARTASS ASSIGNMENT

\documentclass[10pt,fleqn]{article}

%\usepackage{epsf}
\usepackage{amsmath}
\usepackage{amssymb} 
\usepackage{cancel} %cancelled numbers 
%\usepackage{lcg} %random numbers generator
\usepackage{ifthen} %logical constructs
<<<<<<< HEAD
=======
\usepackage{enumerate}
\usepackage{siunitx}

>>>>>>> Added additional default LaTeX packages.

\newif\ifpdf
\ifx\pdfoutput\undefined
\pdffalse
\else
\pdfoutput=1
\pdftrue
\fi

\ifpdf
\usepackage[pdftex]{overpic}
\usepackage[pdftex]{graphicx}
\pdfcompresslevel=9
\pdfpagewidth=215truemm 
\pdfpageheight=297truemm
\pdfhorigin=1truein
\pdfvorigin=1truein
\else
\usepackage{overpic}
\usepackage{graphicx}
\fi

\setlength{\textwidth}{185.0mm}
\setlength{\textheight}{660pt}
\setlength{\oddsidemargin}{-15.0mm}
\setlength{\evensidemargin}{-15.0mm}
\setlength{\headsep}{0pt}
%\setlength{\topmargin}{-29.0truemm}
\setlength{\parindent}{0.0truemm}

\renewcommand {\theenumi}{\bf \arabic{enumi}}
\renewcommand {\theenumii} {\textbf{\arabic {enumii}}}
\renewcommand {\theenumiii} {\textbf {\roman{enumiii}}}


\begin{document}

%%BEGIN DEF 
%definition of variables/modules used
#<
PeriodicCompoundingModule v1;
#>
%%END DEF

%%BEGIN QUESTION

Peter the Mathematician is invited to attend the First International Congress of Mathematical Shoemaking to be held in Hawaii in #<v1.ttime#> time. 
The problem is that Peter would be required to wear shoes but currently he doesn't have any. 
Clever Peter has  a bank account earning #<v1.annualpercent#>\% p.a., compounding #<v1.periodInWords#>.  How much money does he need
to invest now in order to buy a pair of shoes and go to Hawaii, assuming the price of a pair 
of decent shoes is \$#<v1.I#> ?
Ignore fees and taxes, and round your answer appropriately.

%%END QUESTION

%%BEGIN SOLUTION

Let $B$ be the price of the shoes, $I$ be the amount Peter needs to invest, $n$ be the number of compounding
periods before the Congress,
$r$ be the interest compounding #<v1.periodInWords#> .  Then\\[1.8mm]
$r=#<v1.rsolution#>#<v1.rpercent#>$ percent $=#<v1.r#>$, and\\[1.8mm]
$n=#<v1.nsolution#>#<v1.n#>$ \\[1.8mm]
$B=I\left(1+r\right)^{n}$, so $I=\dfrac{B}{\left(1+r\right)^{n}}$. Therefore\\
\begin{eqnarray*}
I&=&\frac{#<v1.I#>}{\left(1+#<v1.r#>\right)^{#<v1.n#>}}\\
&=&\frac{#<v1.I#>}{#<v1.OnePlusrIntoN#>}\\
&\approx&#<v1.decaysol#>
\end{eqnarray*}
Hence he needs to invest approximately \$#<v1.decaysol#> .

%%END SOLUTION

%%BEGIN SHORTANSWER

\$#<v1.decaysol#>

%%END SHORTANSWER
\end{document}
