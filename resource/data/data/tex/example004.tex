%Example SmartAss Template
%created for testing purposes.
%DEFINES THE PREAMBLE FOR STANDARD SMARTASS ASSIGNMENT

\documentclass[10pt,fleqn]{article}

%\usepackage{epsf}
\usepackage{amsmath}
\usepackage{amssymb} 
\usepackage{cancel} %cancelled numbers 
%\usepackage{lcg} %random numbers generator
\usepackage{ifthen} %logical constructs
<<<<<<< HEAD
=======
\usepackage{enumerate}
\usepackage{siunitx}

>>>>>>> Added additional default LaTeX packages.

\newif\ifpdf
\ifx\pdfoutput\undefined
\pdffalse
\else
\pdfoutput=1
\pdftrue
\fi

\ifpdf
\usepackage[pdftex]{overpic}
\usepackage[pdftex]{graphicx}
\pdfcompresslevel=9
\pdfpagewidth=215truemm 
\pdfpageheight=297truemm
\pdfhorigin=1truein
\pdfvorigin=1truein
\else
\usepackage{overpic}
\usepackage{graphicx}
\fi

\setlength{\textwidth}{185.0mm}
\setlength{\textheight}{660pt}
\setlength{\oddsidemargin}{-15.0mm}
\setlength{\evensidemargin}{-15.0mm}
\setlength{\headsep}{0pt}
%\setlength{\topmargin}{-29.0truemm}
\setlength{\parindent}{0.0truemm}

\renewcommand {\theenumi}{\bf \arabic{enumi}}
\renewcommand {\theenumii} {\textbf{\arabic {enumii}}}
\renewcommand {\theenumiii} {\textbf {\roman{enumiii}}}


\begin{document}
%%BEGIN DEF
%Assign any required modules here.
%It is possible to register which 'modules' a particular template uses in the database,
%but I'm not sure what (or if) the table is used. All modules require a default constuctor
%which will generally create random values in required ranges for any parameters. It is 
%also (teorectically) possible to supply initialization parameters by overriding the 
%`initialise` method of `QuestionModule` in your modules.
#<ExampleTemplateModule module#>
%%END DEF
%%BEGIN QUESTION
\begin{enumerate}
\item   First Part: #<module.part001question#>
        Add some question text here that will confuse 
        confound and thoughly test the poor little students
        knowledge. Can refer to `module` also.
\item   Second Part: #<module.part002question#>
        Just for fun, do it again OR indeed as many times
        as you think they can take - maybe one more than
        that.
\end{enumerate}
Hints: (because really we are just big softies!)
\begin{itemize}
\item   Some really useful infomation.
\item   Some more really useful info.
\item   Some really confusing info (we aren't really that nice)
\item   A Joke - because we think it's funny
\end{itemize}
%%END QUESTION
%%BEGIN SOLUTION
\begin{enumerate}
\item   First Part: #<module.part001solution#>
        Clever explanation of how to work out the answer,
        highlighting just how brilliant we are.
\item   Second Part: #<module.part002solution#>
        Because we enjoyed the first part so much
\end{enumerate}
%%END SOLUTION
%%BEGIN SHORTANSWER
\begin{enumerate}
\item   First Part: #<module.part001answer#>
\item   First Part: #<module.part002answer#>
\end{enumerate}
%%END SHORTANSWER
\end{document}
