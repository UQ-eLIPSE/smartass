%Simple question on Avogadro's Law. Given a volume of the first gas v1, find the volume of another gas
%that was produced from the first gas. 
%DEFINES THE PREAMBLE FOR STANDARD SMARTASS ASSIGNMENT

\documentclass[10pt,fleqn]{article}

%\usepackage{epsf}
\usepackage{amsmath}
\usepackage{amssymb} 
\usepackage{cancel} %cancelled numbers 
%\usepackage{lcg} %random numbers generator
\usepackage{ifthen} %logical constructs
<<<<<<< HEAD
=======
\usepackage{enumerate}
\usepackage{siunitx}

>>>>>>> Added additional default LaTeX packages.

\newif\ifpdf
\ifx\pdfoutput\undefined
\pdffalse
\else
\pdfoutput=1
\pdftrue
\fi

\ifpdf
\usepackage[pdftex]{overpic}
\usepackage[pdftex]{graphicx}
\pdfcompresslevel=9
\pdfpagewidth=215truemm 
\pdfpageheight=297truemm
\pdfhorigin=1truein
\pdfvorigin=1truein
\else
\usepackage{overpic}
\usepackage{graphicx}
\fi

\setlength{\textwidth}{185.0mm}
\setlength{\textheight}{660pt}
\setlength{\oddsidemargin}{-15.0mm}
\setlength{\evensidemargin}{-15.0mm}
\setlength{\headsep}{0pt}
%\setlength{\topmargin}{-29.0truemm}
\setlength{\parindent}{0.0truemm}

\renewcommand {\theenumi}{\bf \arabic{enumi}}
\renewcommand {\theenumii} {\textbf{\arabic {enumii}}}
\renewcommand {\theenumiii} {\textbf {\roman{enumiii}}}


\begin{document}
%%BEGIN DEF 
%definition of variables/modules used
%parameters AvogadrosLawModule va(pressure, v1, n1, m1, m2); where pressure - gas pressure,
%           v1 - volume of the first gas,
%           n1 - number of moles of the first gas, 
%           m1 - number of molecules in the formula for the first gas,
%		m2 - number of molecules in the formula for the second gas,
% i.e. for oxigen into ozone 3O2->2O3, m1==3, m2==2
%or different set of parameters va(m1, m2);  where
%  	     m1 - number of molecules in the formula for the first gas,
% 	     m2 - number of molecules in the formula for the second gas,
% i.e. for oxigen into ozone 3O2->2O3, m1==3, m2==2)
%va(1,12.2,0.5,3,2);
#<
AvogadrosLawModule va(3,2);
#>
%%END DEF

%%BEGIN QUESTION

Suppose we have a #<va.v1#> L sample containing #<va.n1#> mol oxygen gas at a pressure of #<va.pressure#> atm.
If all of this oxygen were converted to ozone (O$_{3}$) at the same $T$ and $P$, what would be the volume? 

%%END QUESTION

%%BEGIN SOLUTION
The chemical equation is: $ 3\mbox{O}_{2}(g) \to 2\mbox{O}_{3}(g)$
\vspace*{-3mm}
\begin{eqnarray*}
\mbox{The number of moles O}_{3}\mbox{ produced }&=&#<va.n1#>\mbox{mol O}_{2} \times \dfrac{#<va.m2#>\mbox{mol O}_{3}}{#<va.m1#>\mbox{mol O}_{2}}\\
&=&#<va.n2#>\mbox{mol O}_{3} 
\end{eqnarray*}
\vspace*{-9mm}
\begin{eqnarray*}
\mbox{Since } V/n \mbox{ is constant, } \dfrac{V_{1}}{n_{1}}&=&\dfrac{V_{2}}{n_{2}} \\[2mm]
\therefore V_{2}&=&\dfrac{n_{2}}{n_{1}}\times V_{1}\\[2mm]
&=&\dfrac{#<va.n2#>\mbox{mol}}{#<va.n1#>\mbox{mol}} \times #<va.v1#> \mbox{L}\\[2mm]
&=&#<va.v2#> \mbox{L}
\end{eqnarray*}

%%END SOLUTION

%%BEGIN SHORTANSWER

$#<va.v2#>$L

%%END SHORTANSWER
\end{document}
