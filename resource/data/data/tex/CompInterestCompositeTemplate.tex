% A few simple questions on compounding interest.
%Given initial amount, interest rate, time. Find the final value.
%DEFINES THE PREAMBLE FOR STANDARD SMARTASS ASSIGNMENT

\documentclass[10pt,fleqn]{article}

%\usepackage{epsf}
\usepackage{amsmath}
\usepackage{amssymb} 
\usepackage{cancel} %cancelled numbers 
%\usepackage{lcg} %random numbers generator
\usepackage{ifthen} %logical constructs
<<<<<<< HEAD
=======
\usepackage{enumerate}
\usepackage{siunitx}

>>>>>>> Added additional default LaTeX packages.

\newif\ifpdf
\ifx\pdfoutput\undefined
\pdffalse
\else
\pdfoutput=1
\pdftrue
\fi

\ifpdf
\usepackage[pdftex]{overpic}
\usepackage[pdftex]{graphicx}
\pdfcompresslevel=9
\pdfpagewidth=215truemm 
\pdfpageheight=297truemm
\pdfhorigin=1truein
\pdfvorigin=1truein
\else
\usepackage{overpic}
\usepackage{graphicx}
\fi

\setlength{\textwidth}{185.0mm}
\setlength{\textheight}{660pt}
\setlength{\oddsidemargin}{-15.0mm}
\setlength{\evensidemargin}{-15.0mm}
\setlength{\headsep}{0pt}
%\setlength{\topmargin}{-29.0truemm}
\setlength{\parindent}{0.0truemm}

\renewcommand {\theenumi}{\bf \arabic{enumi}}
\renewcommand {\theenumii} {\textbf{\arabic {enumii}}}
\renewcommand {\theenumiii} {\textbf {\roman{enumiii}}}


\begin{document}
%Parameters could be (initial amount, annual interest rate, number of years, for example,CompoundingCompositeModule v1(100, 0.24, 3);
 
%%BEGIN DEF 
%definition of variables/modules used
#<
CompoundingCompositeModule v1;
#>
%%END DEF

%%BEGIN QUESTION

If \$#<v1.P#> is invested for #<v1.ttime#> at a rate of #<v1.percentannual#>\% per annum, find the final balance if interest compounds: 
\begin{enumerate}
\item
annually?
\item
every six months?
\item
quarterly?
\item
monthly?
\item
continuously?
\end{enumerate}

%%END QUESTION

%%BEGIN SOLUTION

Let $P$ be the amount invested, $r$ be the interest rate per time period, $n$ be the number of time periods and $F$ be the final value.
In each case, $P=#<v1.P#>$. Then:
\begin{enumerate}
\item
Interest compounds annually, so we use the rate and number of time periods given in the question.\\
Hence $r=#<v1.percentannual#>\%=#<v1.rannual#>$ and $n=#<v1.t#>$, so $F=#<v1.P#>\times (1+#<v1.rannual#>)^{#<v1.t#>}=#<v1.P#>\times #<v1.rplus1annual#>^{#<v1.t#>}
#<v1.annualsign#>#<v1.fannual#>$. \\
The final balance is \$$#<v1.fannual#>$.
\item
Interest compounds twice a year, so we need to halve the rate and double the number of time periods given in the question.\\
Hence $r=#<v1.percentsixmonths#>\%=#<v1.rsixmonths#>$ and $n=#<v1.nsixmonths#>$, so $F=#<v1.P#>\times (1+#<v1.rsixmonths#>)^{#<v1.nsixmonths#>}
=#<v1.P#>\times #<v1.rplus1sixmonths#>^{#<v1.nsixmonths#>}#<v1.sixmonthssign#>#<v1.fsixmonths#>$. \\
The final balance is \$$#<v1.fsixmonths#>$.
\item
Interest compounds 4 times a year, so we need to divide the given rate by 4 and multiply the given number of years by 4.\\
Hence $r=#<v1.percentquarterly#>\%=#<v1.rquarterly#>$ and $n=#<v1.nquarterly#>$, so $F=#<v1.P#>\times (1+#<v1.rquarterly#>)^{#<v1.nquarterly#>}
=#<v1.P#>\times #<v1.rplus1quarterly#>^{#<v1.nquarterly#>}#<v1.quarterlysign#>#<v1.fquarterly#>$. \\
The final balance is \$$#<v1.fquarterly#>$.
\item
Interest compounds 12 times a year, so we need to divide the given rate by 12 and multiply the given number of years by 12.\\
Hence $r=#<v1.percentmonthly#>\%=#<v1.rmonthly#>$ and $n=#<v1.nmonthly#>$, so $F=#<v1.P#>\times (1+#<v1.rmonthly#>)^{#<v1.nmonthly#>}
=#<v1.P#>\times #<v1.rplus1monthly#>^{#<v1.nmonthly#>}#<v1.monthlysign#>#<v1.fmonthly#>$. \\
The final balance is \$$#<v1.fmonthly#>$.
\item
Interest compounds continuously, so $F=#<v1.P#>e^{#<v1.rannual#>\times#<v1.t#>}
=#<v1.P#>e^{#<v1.tXr#>}#<v1.contsign#>#<v1.fcont#>$. \\
The final balance is \$$#<v1.fcont#>$.
\end{enumerate}

%%END SOLUTION

%%BEGIN SHORTANSWER

\begin{enumerate}
\item
\$$#<v1.fannual#>$ 
\item
\$$#<v1.fsixmonths#>$
\item
\$$#<v1.fquarterly#>$
\item
\$$#<v1.fmonthly#>$
\item
\$$#<v1.fcont#>$
\end{enumerate}

%%END SHORTANSWER
\end{document}
